%  ========================================================================
%  Copyright (c) 1985 The University of Washington
%
%  Licensed under the Apache License, Version 2.0 (the "License");
%  you may not use this file except in compliance with the License.
%  You may obtain a copy of the License at
%
%      http://www.apache.org/licenses/LICENSE-2.0
%
%  Unless required by applicable law or agreed to in writing, software
%  distributed under the License is distributed on an "AS IS" BASIS,
%  WITHOUT WARRANTIES OR CONDITIONS OF ANY KIND, either express or implied.
%  See the License for the specific language governing permissions and
%  limitations under the License.
%  ========================================================================
%

% Documentation for University of Washington thesis LaTeX document class
% by Jim Fox
% fox@washington.edu
%
%    Revised 2020/02/24, added \caption()[]{} option.  No ToC.
%
%    Revised for version 2015/03/03 of uwthesis.cls
%    Revised, 2016/11/22, for cleanup of sample copyright and title pages
%
%    This document is contained in a single file ONLY because
%    I wanted to be able to distribute it easily.  A real thesis ought
%    to be contained on many files (e.g., one for each chapter, at least).
%
%    To help you identify the files and sections in this large file
%    I use the string '==========' to identify new files.
%
%    To help you ignore the unusual things I do with this sample document
%    I try to use the notation
%       
%    % --- sample stuff only -----
%    special stuff for my document, but you don't need it in your thesis
%    % --- end-of-sample-stuff ---


%    Printed in twoside style now that that's allowed
%
 
\documentclass [11pt, proquest] {uwthesis}[2020/02/24]
 
%
% The following line would print the thesis in a postscript font 

% \usepackage{natbib}
% \def\bibpreamble{\protect\addcontentsline{toc}{chapter}{Bibliography}}

\setcounter{tocdepth}{1}  % Print the chapter and sections to the toc
 

% ==========   Local defs and mods
%

% --- sample stuff only -----
% These format the sample code in this document

\usepackage{alltt}  % 
\newenvironment{demo}
  {\begin{alltt}\leftskip3em
     \def\\{\ttfamily\char`\\}%
     \def\{{\ttfamily\char`\{}%
     \def\}{\ttfamily\char`\}}}
  {\end{alltt}}
 
% metafont font.  If logo not available, use the second form
%
% \font\mffont=logosl10 scaled\magstep1
\let\mffont=\sf
% --- end-of-sample-stuff ---
 



\begin{document}

% ==========   Preliminary pages
%
% ( revised 2012 for electronic submission )
%

\prelimpages

%
% ----- copyright and title pages
%
\Title{The Resolved Evoluton of Nearby Star-forming Galaxies}
\Author{Myles Anthony Aaron Alexander McKay}
\Year{2024}
\Program{Astronomy}

\Chair{Ben Williams}{Professor}{}
\Signature{Mario Juric}
\Signature{Elenor Byler}
\Signature{etc}

\copyrightpage

\titlepage


%
% ----- signature and quoteslip are gone
%

%
% ----- abstract
%


\setcounter{page}{-1}
\abstract{}

%
% ----- contents & etc.
%
\tableofcontents
\listoffigures
%\listoftables  % I have no tables

%
% ----- glossary 
%
\chapter*{Glossary}      % starred form omits the `chapter x'
\addcontentsline{toc}{chapter}{Glossary}
\thispagestyle{plain}
%
\begin{glossary}
  \item[Bolometric Luminosity] The total luminosity radiated by a astronomical source\\ across all wavelengths of electromagnetic spectrum.
  \item[] 


\end{glossary}

%
% ----- acknowledgments
%
\acknowledgments{% \vskip2pc
  % {\narrower\noindent
  The author wishes to express sincere appreciation to
  University of Washington, where he has had the opportunity
  to work with the \TeX\ formatting system,
  and to the author of \TeX, Donald Knuth, {\it il miglior fabbro}.
  % \par}
}

%
% ----- dedication
%
\dedication{}

%
% end of the preliminary pages



%
% ==========      Text pages
%

\textpages

% ========== Chapter 1

\chapter {Introduction}




\section{The Purpose of This Sample Thesis}




\section{Conventions and Notations}




\section{Nota bene}




% ========== Chapter 2

\chapter{A Brief \\ Description of \protect\TeX}


\section{What is it; why is it spelled that way;
  and what do
  really long section titles look like in the text and in the
  Table of Contents?}


\section{\TeX books}



\section{Mathematics}





\section{Languages other than English}



% ========== Chapter 3

\chapter{The Thesis Unformatted}



\section{The Control File}



\section{The Text Pages}




\subsection{Footnotes}

\label{footnotes}


\subsection{Figures and Tables}


\subsubsection{Facing pages}






\subsection{Horizontal Figures and Tables}




\subsection{Figure and Table Captions}



\subsection{Line spacing}


\section{The Preliminary Pages}



\subsection{Copyright page}
Print the copyright page with \verb"\copyrightpage".

\subsection{Title page}
Print the title page with \verb"\titlepage".
The title page of this thesis was printed with%

\begin{demo}
  \\titlepage
\end{demo}



% \subsection{Abstract}
% Print the
% abstract with \verb"\abstract".
% It has one argument, which is the text of the abstract.
% All the names have already been defined.
% The abstract of this thesis was printed with

% \begin{demo}
%   \\abstract\{This sample . . . `real' dissertation.\}
% \end{demo}


% \subsection{Tables of contents}
% Use the standard \LaTeX\ commands to format these items.


% \subsection{Acknowledgments}
% Use the \verb"\acknowledgments" macro to format the acknowledgments page.
% It has one argument, which is the text of the acknowledgment.
% The acknowledgments of this thesis was printed with

% \begin{demo}
%   \\acknowledgments\{The author wishes . . . \{\\it il miglior fabbro\}.\\par\}\}
% \end{demo}


% \subsection{Dedication}


% \subsection{Vita}



%%  
%% \section{Customization of the Macros}
%%  
%% Simple customization, including 
%% alteration of default parameters,  changes to dimensions,
%% paragraph indentation, and margins, are not too difficult.
%% You have the choice of modifying the class file ({\tt uwthesis.cls})
%% or loading
%% one or more personal style files to customize your thesis.
%% The latter is usually most convenient, since you do not need
%% to edit the large and complicated class file.
%% 



% ========== Chapter 4

\chapter{Running \LaTeX\\
  ({\it and printing if you must})}


From a given source \TeX\ will produce exactly the same document
on all computers and, if needed, on all printers.  {\it Exactly the same}
means that the various spacings, line and page breaks, and
even hyphenations will occur at the same places.

How you edit your text files and run \LaTeX\ varies
from system to system and depends on your personal preference.

\section{Running}

The author is woefully out of his depth where
\TeX\ on Windows is concerned.  Google would be his resource.
On a linux system he types

\begin{demo}
  \$\ pdflatex uwthesis
\end{demo}

and it generally works.


\section{Printing}

All implementations of \TeX\ provide the option of {\bf pdf} output,
which is all the Graduate School requires.  Even if you intend to
print a copy of your thesis create a
  {\tt pdf}.  It will print most anywhere.

\printendnotes

%
% ==========   Bibliography
%
\nocite{*}   % include everything in the uwthesis.bib file
\bibliographystyle{plain}
\bibliography{uwthesis}
%
% ==========   Appendices
%
\appendix
\raggedbottom\sloppy

% ========== Appendix A

\chapter{Where to find the files}

The uwthesis class file, {\tt uwthesis.cls}, contains the parameter settings,
macro definitions, and other \TeX nical commands which
allow \LaTeX\ to format a thesis.
The source to
the document you are reading, {\tt uwthesis.tex},
contains many formatting examples
which you may find useful.
The bibliography database, {\tt uwthesis.bib}, contains instructions
to BibTeX to create and format the bibliography.
You can find the latest of these files on:

\begin{itemize}
  \item My page.
        \begin{description}
          \item[] \verb%https://staff.washington.edu/fox/tex/thesis.shtml%
        \end{description}

  \item CTAN
        \begin{description}
          \item[]  \verb%http://tug.ctan.org/tex-archive/macros/latex/contrib/uwthesis/%
          \item[]  (not always as up-to-date as my site)
        \end{description}

\end{itemize}

\vita{Jim Fox is a Software Engineer with IT Infrastructure Division at the University of Washington.
  His duties do not include maintaining this package.  That is rather
  an avocation which he enjoys as time and circumstance allow.

  He welcomes your comments to {\tt fox@uw.edu}.
}


\end{document}
